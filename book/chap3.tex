\chapter{关于虚拟化的一段对话}
\thispagestyle{empty}

\textbf{教授:}首先我们来看操作系统三条简洁之道的第一个: 虚拟化。\newline
\textbf{学生:}但是什么是虚拟化呢? 尊敬的教授。\newline
\textbf{教授:}假设我们有一个桃子。\newline
\textbf{学生:}一个桃子?(怀疑)\newline
\textbf{教授:}是的, 一个桃子。让我们把它叫做实物桃。 但是我们有许多食客想吃这个桃子。我们想向每个食客展示这就是他们的自己的桃子, 这样他们才会高兴, 我们把这个桃子叫做虚拟桃。我们用某种方法让一个实物桃变成若干个虚拟桃。重要的是: 在这种错觉中, 看起来就好像每个人都有一个实物桃, 尽管并不是这样。\newline
\textbf{学生:}所以其实你在分享桃子,但你压根不知道?\newline
\textbf{教授:}是的!\newline
\textbf{学生:}但是这样还是只会有一颗实物桃子。\newline
\textbf{教授:}对的。所以...?\newline
\textbf{学生:}Emmm, 如果我和别人分享桃子,我想我会发现的。\newline
\textbf{教授:}没错!但是大部分时间他们会打盹或在做一些其他事情,这样,我们就可以把桃拿走然后给其他人一会, 这样我们就创造出一种每个人都有一个桃的幻觉\newline
\textbf{学生:}听起来像是个糟糕的竞选口号。你说的是有关计算机的,对吧教授?\newline
\textbf{教授:}噢,你希望有一个更切实的例子。好主意!让我们以最基本的资源:CPU为例。虚拟化所做的就是用这单个CPU,让系统上运行的应用程序看起来有许多虚拟CPU。这样,每个应用程序都认为有属于自己的CPU可以使用,尽管只有一个物理CPU,因此操作系统就创造了一个美丽的幻觉:它虚拟化了CPU。\newline
\textbf{学生:}哇!这听起来就像是魔术,快告诉我更多,它是如何做到的。\newline
\textbf{教授:}听起来你们准备好开始了。\newline
\textbf{学生:}是的!好吧,有一点我必须承认,我有点担心你又开始讨论桃子\newline
\textbf{教授:}别担心;我甚至压根不喜欢桃子,So,我们开始吧...\newline